\documentclass{beamer}
\usepackage{fancyvrb}

\newcommand\slide[2]{
\begin{frame}[fragile=singleslide]{#1}
#2
\end{frame}
}

\newcommand\cmd[1]{{\tt \textbackslash #1}}

\mode<presentation>
{
  \usetheme{Berkeley}
}

\title{\LaTeX~tutorial}
\author{I.~Holmes}
\institute{
  Department of Bioengineering\\
  University of California, Berkeley}

\begin{document}

%%%
\begin{frame}
  \titlepage
\end{frame}

%%%
\slide{Outline}{\tableofcontents}

\section{Why \LaTeX?}

%%%
\slide{\LaTeX~advocacy}{
  \begin{enumerate}
    \item It's free, portable, open source \& extensible
    \item Source files are plain text, revision control easier
    \item Typesetting is {\em much} better, especially math
    \item Style changes are easier
    \item {\bf Separation of form and content}
  \end{enumerate}
}

%%%
\slide{\LaTeX~criticism}{
  \begin{enumerate}
    \item It's the worst programming language ever
    \item Many things you just can't do
    \item Syntax is horrible
    \item Compilation from source is almost impossible
    \item WYSIWYG editors suck
  \end{enumerate}
}

\section{Hello World}

%%%
\begin{frame}[fragile=singleslide]{helloworld.tex}
\begin{Verbatim}[frame=single]
\documentclass{article}
\title{Marmosets Are Great}
\author{Ian Holmes}

\begin{document}
\maketitle
\abstract{A short treatise on marmosets.}

\section{Introduction}
Marmosets ({\em Callitrichidae})
are {\bf New World Monkeys}.

\end{document}
\end{Verbatim}
Compile with {\tt pdflatex helloworld.tex}
\end{frame}

%%%
\begin{frame}[fragile=singleslide]{Section references}
Can use \cmd{label} and \cmd{ref} as follows:
\begin{Verbatim}[frame=single]
\section{Introduction}
\label{intro}
Marmosets are New World Monkeys.

\section{Geography}
Marmosets are found in the New World,
as mentioned in Section~\ref{intro}.
\end{Verbatim}
Note tilde {\tt \~{}} between {\tt Section} and \cmd{ref}: prevents linebreak.
}

\section{Makefiles}

%%%
\begin{frame}[fragile=singleslide]{Makefiles}
If you change the section numbers, you will have to re-run {\tt pdflatex}.
Consequently, it's common to run the program twice.
Can do this with a Makefile:
\begin{Verbatim}[frame=single]
helloworld.pdf: helloworld.tex
	pdflatex helloworld.tex
	pdflatex helloworld.tex
\end{Verbatim}

General form of Makefile stanza:
\begin{Verbatim}[frame=single]
TARGET: DEPENDENCIES
  <TAB> COMMANDS
\end{Verbatim}
\end{frame}

%%%
\begin{frame}[fragile=singleslide]{Make command-line usage}
\begin{itemize}
\item General: {\tt make helloworld.pdf}
\item Force rebuild: {\tt make -B helloworld.pdf}
\item Dry run: {\tt make -n helloworld.pdf}
\end{itemize}

By default, {\tt make} just builds first target in Makefile.
\end{frame}

%%%
\begin{frame}[fragile=singleslide]{Makefiles and replicability}

Titus Brown's checklist for paper replicability:
\begin{itemize}
\item a link to the paper itself, in preprint form, stored at arXiv;
\item a tutorial for running the software on a Linux machine hosted in the Amazon cloud;
\item a git repository for the software itself (hosted on github);
\item a git repository for the LaTeX paper and analysis scripts, including an ipython notebook for generating the figures;
\item instructions on how to start up an EC2 cloud instance, install the software and paper pipeline, and build most of the analyses and all of the figures from scratch;
\item the data necessary to run the pipeline;
\item some of the output data discussed in the paper.
\end{itemize}
\url{http://ivory.idyll.org/blog/replication-i.html}

\end{frame}

%%%
\begin{frame}[fragile=singleslide]{Pseudotargets and variables}
\begin{Verbatim}[frame=single]
MAIN = helloworld

all: $(MAIN).pdf

%.pdf: %.tex
	pdflatex $<
	pdflatex $<
        open $@

clean:
	rm *.toc *.log *.out *.pdf *.aux *~
\end{Verbatim}
\end{frame}


\end{document}
