\documentclass{beamer}
\usepackage{fancyvrb}
\usepackage{url}
\usepackage{color}
\usepackage[normalem]{ulem}
\usepackage{soul}

\newcommand\slide[2]{
\begin{frame}[fragile=singleslide]{#1}
#2
\end{frame}
}

\newcommand\cmd[1]{{\tt \textbackslash #1}}

\newcommand\ex[1]{\colorbox{red}{\color{white} #1}}
\newcommand\trythis{\ex{Try this.}\ }
\newcommand\shell[1]{\colorbox{black}{\color{green} \tt #1}}
\newcommand\sample[1]{{\tt #1}}
\newcommand\filename[1]{{\tt #1}}

\DeclareUrlCommand\curl{%
  \renewcommand\UrlFont{\ttfamily\color{blue}}%
  \renewcommand\UrlLeft{\uline\bgroup}%
  \renewcommand\UrlRight{\egroup}}


\mode<presentation>
{
  \usetheme{Berkeley}
}

\title{\LaTeX~tutorial}
\author{I.~Holmes}
\institute{
  Department of Bioengineering\\
  University of California, Berkeley}

\begin{document}

%%%
\begin{frame}
  \titlepage
\end{frame}

%%%
\slide{Outline}{\tableofcontents}

%%%
\begin{frame}[fragile=singleslide]{Key}
\begin{enumerate}
\item \ex{Exercises in white on red}
\item \shell{Shell commands in green type on black}
\item URLs in blue: \curl{http://tinyurl.com/texroll}
\item Source code in boxes
\end{enumerate}
\begin{Verbatim}[frame=single]
This is some LaTeX source code
\end{Verbatim}

\ex{Try the URL now!}
}


\section{Why \LaTeX?}

%%%
\slide{\LaTeX~advocacy}{
  \begin{enumerate}
    \item It's free, portable, open source \& extensible
    \item Source files are plain text, revision control easier
    \item Typesetting is {\em much} better, especially math
    \item Style changes are easier
    \item Easy to integrate with programmatic workflows
    \item {\bf Separation of form and content}
  \end{enumerate}
}

%%%
\slide{\LaTeX~criticism}{
  \begin{enumerate}
    \item Possibly the worst programming language ever
    \item Syntax is horrible
    \item Compilation from source is almost impossible
    \item Mostly trial and error, unless you're a guru
    \item Some things you just can't do (unless, guru)
    \item {\bf Will mark you forever as a nerd pariah}
  \end{enumerate}
}

\section{Hello World}

%%%
\slide{Text editor}{
  Before you start, your life will be much easier with
  a text editor that has \LaTeX\ syntax-coloring,
  such as \shell{emacs} or \shell{vim}.

  \vspace{\baselineskip}
  You could also use a specialized \LaTeX\ editor
  that gives you previews, such as TeXmaker,
  or even a WYSIWYG \LaTeX\ editor such as LyX (free)
  or Texpad (OSX, \$\$\$).

  \vspace{\baselineskip}
  Best of all is \curl{https://www.overleaf.com/} ---
  collaborative browser-based editor. Try it!

  \vspace{\baselineskip}
  You still need to understand the \LaTeX\ underneath.
  So...
}

%%%
\begin{frame}[fragile=singleslide]{\filename{helloworld.tex}}
\begin{Verbatim}[frame=single]
\documentclass{article}
\title{Marmosets Are Great}
\author{Ian Holmes}
\begin{document}
\maketitle
\abstract{A short treatise on marmosets.}
\section{Introduction}
Marmosets ({\em Callitrichidae})
are {\bf New World Monkeys}.
\end{document}
\end{Verbatim}
Compile with \shell{pdflatex helloworld.tex}

\trythis \hspace{\fill} \curl{http://tinyurl.com/texhello}
\end{frame}

%%%
\begin{frame}[fragile=singleslide]{\sample{documentclass} options}
\begin{itemize}
\item \cmd{documentclass[10pt]\{article\}}
\item \cmd{documentclass[twocolumn]\{article\}}
\item \cmd{documentclass[landscape]\{article\}}
\item \cmd{documentclass\{letter\}}
\item \cmd{documentclass\{book\}}
\item \cmd{documentclass\{beamer\}} --- presentation
\end{itemize}

Further classes can be defined using a {\em class file}.

For example, the journal {\em Bioinformatics} provides a class file \filename{bioinfo.cls}
invoked with \cmd{documentclass\{bioinfo\}}.

\end{frame}

%%%
\begin{frame}[fragile=singleslide]{Section references}
Can use \cmd{label} and \cmd{ref} as follows:
\begin{Verbatim}[frame=single]
\section{Introduction}
\label{intro}
Marmosets are New World Monkeys.

\section{Geography}
Marmosets are found in the New World,
as mentioned in Section~\ref{intro}.
\end{Verbatim}
Note tilde \sample{\~{}} between \sample{Section} and \cmd{ref}: prevents linebreak.

\ex{Add a section or two, and recompile.}
}

\section{Makefiles}

%%%
\begin{frame}[fragile=singleslide]{Makefiles}
If you change the section numbers, you will have to re-run \shell{pdflatex}.
Consequently, it's common to run the program twice.
Can do this with a \filename{Makefile}:
\begin{Verbatim}[frame=single]
helloworld.pdf: helloworld.tex
	pdflatex helloworld.tex
	pdflatex helloworld.tex
\end{Verbatim}

General form of Makefile stanza:
\begin{Verbatim}[frame=single]
TARGET: DEPENDENCIES
  <TAB> COMMANDS
\end{Verbatim}
\end{frame}

%%%
\begin{frame}[fragile=singleslide]{Make command-line usage}
\begin{itemize}
\item General: \shell{make helloworld.pdf}
\item Force rebuild: \shell{make -B helloworld.pdf}
\item Dry run: \shell{make -n helloworld.pdf}
\end{itemize}

By default, \shell{make} just builds first target in Makefile.
\end{frame}

%%%
\begin{frame}[fragile=singleslide]{Makefiles and replicability}

Titus Brown's checklist for paper replicability:
\begin{itemize}
\item a link to the paper itself, in preprint form, stored at arXiv;
\item a tutorial for running the software on a Linux machine hosted in the Amazon cloud;
\item a git repository for the software itself (hosted on github);
\item a git repository for the LaTeX paper and analysis scripts, including an ipython notebook for generating the figures;
\item instructions on how to start up an EC2 cloud instance, install the software and paper pipeline, and build most of the analyses and all of the figures from scratch;
\item the data necessary to run the pipeline;
\item some of the output data discussed in the paper.
\end{itemize}
\curl{http://ivory.idyll.org/blog/replication-i.html}

\end{frame}

%%%
\begin{frame}[fragile=singleslide]{Pseudotargets, pattern rules and variables}
\begin{Verbatim}[frame=single]
MAIN = helloworld

all: $(MAIN).pdf

%.pdf: %.tex
	pdflatex $<
	pdflatex $<
        open $@

clean:
	rm *.toc *.log *.out *.pdf *.aux *~
\end{Verbatim}

\trythis
\quad
\parbox{.6\textwidth}{
Use \shell{make}, \shell{make -n} and \shell{make -B}.

If in doubt: \shell{make clean}
}
\end{frame}

%%%
\slide{Loading other files}{
  \LaTeX\ files can include other files via the \cmd{input} command.
  This is particularly useful with Makefiles, because you can generate
  data-driven parts of your article automatically, and combine them
  with manually-written sections.

  \vspace{\baselineskip}
  The \cmd{include} command is like \cmd{input} but does some extra book-keeping
  (such as adding a page break). Useful for e.g. separating a thesis into chapter files.
}

\begin{frame}[fragile=singleslide]{Loading other files (example)}

\begin{Verbatim}[frame=single]
\documentclass[dvips,12pt]{book}
\usepackage{color,graphics,palatino}
\begin{document}
\pagestyle{empty}  % page numbers off
\tableofcontents
\listoffigures
\pagestyle{plain}  % page numbers on
\input{chapter1}   % includes `chapter1.tex'
\input{chapter2}   % includes `chapter2.tex'
\input{fig2}       % includes `fig2.tex'
\input{chapter3}   % includes `chapter3.tex'
\input{chapter4}   % includes `chapter4.tex'
\appendix
\input{appendices} % includes `appendices.tex'
\end{document}
\end{Verbatim}

}

\section{Styling}

%%%
\begin{frame}[fragile=singleslide]{Comments, escapes, styling}
\begin{Verbatim}[frame=single]
% Comments
Actual percent sign: 100\%
Other escapes: \_, \&
Tilde escape: \~{}

``Pretty quotation marks''

Empty line signals new paragraph.

Space: \quad Explicit line \\ break
\end{Verbatim}

Can you get {\bf bold}, {\em italic} \& {\tt typewriter} fonts?

\ex{Google these typefaces.}
\end{frame}

%%%
\slide{Typefaces}{
\begin{tabular}{ll}
\sample{\{\cmd{bf} Bold\}} & {\bf Bold} \\
\sample{\{\cmd{em} Italic\}} & {\em Italic} \\
\sample{\{\cmd{tt} Typewriter\}} & {\tt Typewriter}
\end{tabular}
}

%%%
\slide{Page numbering}{
These go in the preamble:
\begin{itemize}
\item \cmd{pagenumbering\{arabic\}} --- default
\item \cmd{pagenumbering\{roman\}}
\item \cmd{pagenumbering\{Roman\}}
\item \cmd{pagenumbering\{alph\}}
\item \cmd{pagenumbering\{Alph\}}
\end{itemize}

\vspace{\baselineskip}
To suppress page numbers altogether, use \cmd{pagestyle\{empty\}}.

\vspace{\baselineskip}
To add a table of contents: \cmd{tableofcontents}
}

%%%
\begin{frame}[fragile=singleslide]{Lists}
\begin{Verbatim}[frame=single]
List of books about wizard school
\begin{itemize}
\item Earthsea
\item Harry Potter
\item The Magicians
\item The Once and Future King
\end{itemize}
\end{Verbatim}
\trythis

Also try \sample{enumerate} instead of \sample{itemize},
and try nesting lists inside other lists.

\ex{How many levels deep can you nest?}
\end{frame}

%%%
\begin{frame}[fragile=singleslide]{Tables}
\begin{tabular}{rcl}
Right-justified & Centered & Left-justified \\
\hline
School vouchers & Science & Public education \\
Defense spending & Trade deals & Aid programs
\end{tabular}

\begin{Verbatim}[frame=single]
\begin{tabular}{rcl}
Right-justified & Centered & Left-justified \\
\hline
School vouchers & Science & Public education \\
Defense spending & Trade deals & Aid programs
\end{tabular}
\end{Verbatim}

\ex{Time to add a table.}

Make a table with some facts about marmosets.
Or pick another vertebrate from \curl{hgdownload.cse.ucsc.edu}
and make a table about it.
\end{frame}

%%%
\begin{frame}[fragile=singleslide]{Table captions and references}
\begin{Verbatim}[frame=single]
\begin{table}
\begin{tabular}
...
\end{tabular}
\caption{
 \label{MarmosetFacts}
 A table of marmoset facts.
}
\end{table}

For marmoset data, see Table~\ref{MarmosetFacts}.
\end{Verbatim}
\end{frame}

\section{Mathematics}

%%%
\begin{frame}[fragile=singleslide]{Equations}
\begin{Verbatim}[frame=single]
Inline: $a = 3$, $b = 5$

Non-numbered:
\[
y = ax + b
\]

\end{Verbatim}

Inline: $a = 3$, $b = 5$

Non-numbered:
\[
y = ax + b
\]
\end{frame}



%%%
\begin{frame}[fragile=singleslide]{Equations}
\begin{Verbatim}[frame=single]
Numbered (Equation~\ref{Gaussian}):
\begin{equation}
x \sim {\cal N}(\mu,\sigma):
\quad
P(x' \leq x < x' + dx') =
\frac{1}{\sqrt{2 \pi \sigma^2}}
e^{-\frac{(x'-\mu)^2}{2\sigma^2}} dx'
\label{Gaussian}
\end{equation}
\end{Verbatim}

Numbered (Equation~\ref{Gaussian}):
\begin{equation}
x \sim {\cal N}(\mu,\sigma):
\quad
P(x' \leq x < x' + dx') =
\frac{1}{\sqrt{2 \pi \sigma^2}}
e^{-\frac{(x'-\mu)^2}{2\sigma^2}} dx'
\label{Gaussian}
\end{equation}

\end{frame}

%%%
\begin{frame}[fragile=singleslide]{Equations}
\begin{Verbatim}[frame=single]
Numbered (Equation~\ref{Gaussian}):
\begin{equation}
x \sim {\cal N}(\mu,\sigma):
\quad
P(x' \leq x < x' + dx') =
\frac{1}{\sqrt{2 \pi \sigma^2}}
e^{-\frac{(x'-\mu)^2}{2\sigma^2}} dx'
\label{Gaussian}
\end{equation}
\end{Verbatim}

\ex{Google ``Latex math symbols''.}

\ex{Write out another distribution e.g. Poisson.}
\end{frame}


%%%
\begin{frame}[fragile=singleslide]{Brackets, arrays}
\begin{Verbatim}[frame=single]
\[
\left(
 \begin{array}{c}
  n \\
  k
 \end{array}
 \right)
= \frac{n \times (n-1) \ldots \times (n-k+1)}
       {k \times (k-1) \ldots \times 2 \times 1}
= \frac{n!}{k!(n-k)!}
\]
\end{Verbatim}

\[
\left(
 \begin{array}{c}
  n \\
  k
 \end{array}
\right)
= \frac{n \times (n-1) \ldots \times (n-k+1)}
       {k \times (k-1) \ldots \times 2 \times 1}
= \frac{n!}{k!(n-k)!}
\]
\end{frame}


%%%
\begin{frame}[fragile=singleslide]{Macro commands}

\begin{Verbatim}[frame=single]
\newcommand\MacroName[NumberOfArgs]{
  Definition of macro
  (Arg1 is #1, Arg2 is #2...)
}
\end{Verbatim}

e.g.
  
\begin{Verbatim}[frame=single]
\newcommand\isa[2]{
 \fbox{ A {\em #1} is a type of {\em #2}. }
}
\isa{marmoset}{mammal}
\isa{mammal}{verterbrate}
\end{Verbatim}

\vspace{\baselineskip}
  \newcommand\isa[2]{\fbox{ A {\em #1} is a type of {\em #2}. } }
  \isa{marmoset}{mammal}
  \isa{mammal}{verterbrate}
\end{frame}

%%%
\begin{frame}[fragile=singleslide]{Macro commands}
\begin{Verbatim}[frame=single]
\newcommand\binomial[2]{
\left(
 \begin{array}{c}
  #1 \\
  #2
 \end{array}
 \right)
}
\[
 \binomial{5}{2} = (5 \times 4) / 2 = 10
\]
\end{Verbatim}

\newcommand\binomial[2]{
\left(
 \begin{array}{c}
  #1 \\
  #2
 \end{array}
 \right)
}
$\binomial{5}{2} = (5 \times 4) / 2 = 10$
\quad \trythis
\end{frame}


%%%
\begin{frame}[fragile=singleslide]{More arrays; text in math environments}
\begin{Verbatim}[frame=single]
\[
H(x) = \left\{
 \begin{array}{ll}
  0 & \mbox{for $x < 0$} \\
  1 & \mbox{for $x \geq 0$}
 \end{array}
\right.
\]
\end{Verbatim}

\[
H(x) = \left\{
 \begin{array}{ll}
  0 & \mbox{for $x < 0$} \\
  1 & \mbox{for $x \geq 0$}
 \end{array}
\right.
\]
\end{frame}


%%%
\begin{frame}[fragile=singleslide]{Equation arrays}
\begin{Verbatim}[frame=single]
\begin{eqnarray}
F_1 & = & 1 \\
F_2 & = & 1 \\
F_{n+2} & = & F_n + F_{n+1}
\end{eqnarray}
\end{Verbatim}
\begin{eqnarray}
F_1 & = & 1 \\
F_2 & = & 1 \\
F_{n+2} & = & F_n + F_{n+1}
\end{eqnarray}
\end{frame}

%%%
\begin{frame}[fragile=singleslide]{Equation arrays (cleaner numbering)}
\begin{Verbatim}[frame=single]
\begin{eqnarray}
F_1 & = & 1 \nonumber \\
F_2 & = & 1 \nonumber \\
F_{n+2} & = & F_n + F_{n+1}
\label{Fibonacci}
\end{eqnarray}
Fibonacci numbers (\ref{Fibonacci})
arise naturally in phyllotaxis.
\end{Verbatim}
\begin{eqnarray}
F_1 & = & 1 \nonumber \\
F_2 & = & 1 \nonumber \\
F_{n+2} & = & F_n + F_{n+1}
\label{Fibonacci}
\end{eqnarray}
Fibonacci numbers (\ref{Fibonacci}) arise naturally in phyllotaxis.
\end{frame}


%%%
\begin{frame}[fragile=singleslide]{Equation arrays (no numbering)}
\begin{Verbatim}[frame=single]
\begin{eqnarray*}
F_1 & = & 1 \\
F_2 & = & 1 \\
F_{n+2} & = & F_n + F_{n+1}
\end{eqnarray*}
\end{Verbatim}
\begin{eqnarray*}
F_1 & = & 1 \\
F_2 & = & 1 \\
F_{n+2} & = & F_n + F_{n+1}
\end{eqnarray*}
\end{frame}

\section{Marmosets}

%%%
\slide{Interlude}{
\ex{Do the following:}
\vspace{\baselineskip}
\begin{enumerate}
\item Download a set of predicted gene annotations from UCSC for your
  vertebrate of choice. (I used the Augustus gene predictions for
  marmoset.)
\item Also download the description of that table. Find out which
  column in the table has the number of exons for each gene.
\item Using \shell{perl}, \shell{python}, \shell{sed}, \shell{cut},
  or another such tool, extract the number of exons as a column of
  numbers.
\item Plot the frequency distribution in R (or otherwise).
\item Export as a PDF file, \filename{exonFreqs.pdf}
\end{enumerate}
}

%%%
\begin{frame}[fragile=singleslide]{Example R script}
\begin{Verbatim}[frame=single]
require("ggplot2")
exons <- read.csv("numExons.txt")
plot <- qplot(exons,geom="histogram",binwidth=1,
       main="Marmoset exon count distribution") +
       scale_y_continuous() +
       scale_x_continuous(limits=c(0,100)) +
       xlab("Number of exons")
ggsave ("exonFreqs.pdf", plot)
\end{Verbatim}

Save as \filename{plot.R}
\end{frame}


%%%
\begin{frame}[fragile=singleslide]{Example Makefile}
\begin{Verbatim}[frame=single]
PREFIX := hgdownload.cse.ucsc.edu/goldenPath
SPECIES := calJac3
URL := http://$(PREFIX)/$(SPECIES)/database

augustusGene.sql augustusGene.txt.gz:
	curl -O $(URL)/$@

%.txt: %.txt.gz
	gunzip --keep $<

numExons.txt: augustusGene.txt
	cat $< | cut -f 9 >$@

exonFreqs.pdf: numExons.txt plot.R
	R -f plot.R
\end{Verbatim}
\end{frame}

% just to keep emacs syntax highlighting happy:
%$

\section{Figures}

%%%
\begin{frame}[fragile=singleslide]{Figure}
\begin{Verbatim}[frame=single]
\includegraphics{numExons.pdf}
\end{Verbatim}

Of course, you can get more elaborate...
\end{frame}

%%%
\begin{frame}[fragile=singleslide]{Figure, with caption}
\begin{Verbatim}[frame=single]
\begin{figure}
\includegraphics[width=\textwidth]{numExons.pdf}
\caption{
  \label{ExonDistribution}
  Distribution of exon frequencies in marmosets.
}
\end{figure}
\end{Verbatim}
\end{frame}

\section{Bibliography}

%%%
\begin{frame}[fragile=singleslide]{BibTeX}
\begin{Verbatim}[frame=single]
Marmosets are highly social \cite{Marx2016}.
...
\bibliographystyle{natbib}  % or plain, unsrt, ...
\bibliography{references}
\end{Verbatim}

Implies the existence of a file \filename{references.bib}
\begin{Verbatim}[frame=single]
@Article{Marx2016,
 Author="Marx, V.",
 Title="{N}eurobiology: learning from marmosets",
 Journal="Nat. Methods",
 Year="2016",
 Volume="13",
 Number="11"
}
\end{Verbatim}
\end{frame}

%%%
\begin{frame}[fragile=singleslide]{Running BibTeX}
Typically you need to run \shell{pdflatex}, then \shell{bibtex}, then \shell{pdflatex} again {\em twice} to ensure all numbering is correct.

\vspace{\baselineskip}
In your \filename{Makefile}:

\vspace{\baselineskip}
\begin{Verbatim}[frame=single]
%.pdf: %.tex references.bib
	pdflatex $<
        bibtex $<
	pdflatex $<
	pdflatex $<
\end{Verbatim}

\vspace{\baselineskip}

{\em Yes: this is really messed-up}

\end{frame}

%%%
\slide{TeXMed}{
\curl{http://www.bioinformatics.org/texmed/}

BibTeX wrapper for PubMed.

\vspace{\baselineskip}

\ex{Try adding a reference for your vertebrate of choice.}
}

\section{Advanced}

%%%
\slide{Commands}{
  \begin{tabular}{ll}
    Command & Purpose \\
    \hline
    \cmd{hspace} & Fill horizontal space \\
    \cmd{fbox} & Box with frame \\
    \cmd{parbox} & Box with line breaks \\
    \cmd{newcounter} & Create a new counter \\
    \cmd{stepcounter} & Increment counter \\
    \cmd{color} & Change text color \\
    \cmd{colorbox} & Change background color \\
  \end{tabular}

\vspace{\baselineskip}

See e.g. \curl{https://en.wikibooks.org/wiki/LaTeX}
}
  
%%%
\slide{Packages}{
  Loaded with \cmd{usepackage}, e.g. \cmd{usepackage\{amsmath\}}

  \vspace{.75\baselineskip}
  \begin{tabular}{ll}
    Package & Purpose \\
    \hline
    {\tt algorithm2e} & Writing out algorithms \\
    {\tt beamer} & Presentations (like this one) \\
    {\tt amsmath} & Better math formatting \\
    {\tt geometry} & Page formatting (e.g. margins) \\
    {\tt biblatex} & Better bibliographies \\
    {\tt chemfig} & Chemical structures \\
  \end{tabular}

\vspace{.75\baselineskip}
\centerline{ \ex{Try a few of these out...} }

\vspace{.75\baselineskip}
CTAN (\curl{ctan.org}): Comprehensive TeX Archive Network

\vspace{.75\baselineskip}
These slides at \curl{https://github.com/ihh/latex-tutorial}
}

\slide{Homework}{
Make a short report on a vertebrate in the UCSC genome database that is {\em not} a marmoset.
Include:
\begin{enumerate}
\item Title, author, abstract
\item Two-column layout
\item Introduction, Results, References sections
\item A figure showing the distribution of exon counts (or other data from UCSC)
\item A mathematical formula (e.g. a fit to the histogram)
\item At least one table
\item At least one reference
\end{enumerate}
}

\end{document}
